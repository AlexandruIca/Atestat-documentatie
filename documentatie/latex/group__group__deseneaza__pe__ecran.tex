\hypertarget{group__group__deseneaza__pe__ecran}{}\section{Desenand pe ecran}
\label{group__group__deseneaza__pe__ecran}\index{Desenand pe ecran@{Desenand pe ecran}}
In primul rand pentru a putea desena orice pe ecran in c++(si nu numai) trebuie sa creem o fereastra. Procesul de a crea o fereastra este specific fiecraui sistem de operare. In windows avem la dispozitie functia Create\+Window\+Ex, in lumea Linux poate fi creata cu ajutorul Xlib/\+X\+CB sau Wayland, iar pe Mac OS o varianta ar fi Cocoa(desi nu este nativ c++ poate fi creat un wrapper cu ajutorul llvm-\/clang). Jocul meu suporta toate aceste platforme(poate chiar si android si ios) si totusi nu trebuie sa contina cod pentru fiecare platforma in parte. Asta datorita faptului ca foloseste o librarie numita S\+DL care a facut deja asa ceva. Libraria este inclusa in proiect. In plus imi ofera si functii pentru a desena imagini si figuri geometrice simple. Daca nu m-\/as fi folosit de o astfel de librarie ar fi trebuit sa folosesc opengl care este suportat pe multe platforme(dar de ceva timp nu mai este suportat pe ios is Mac O\+S) sau sa recurg la mai mult cod care depinde de platforme(direct3d -\/ windows, Metal/\+Quartz -\/ Mac\+Os/ios etc.).

Codul care deschide o fereastra se gaseste in functia \hyperlink{main_8cpp_a81cdc1223b468897943076b72f048133}{initializeaza}. Prima data chem functia S\+D\+L\+\_\+\+Init care va returna o valoarea negativa daca va esua, apoi creez efectiv fereastra cu ajutorul S\+D\+L\+\_\+\+Create\+Window. Primul parametru este titlul ferestrei, al doilea si al treilea reprezinta coordonatele unde va fi afisata fereastra, al patrulea si al cincilea reprezinta lungime si inaltimea ferestrei, iar ultimul nu este atat de important.

Odata ce am deschis fereastra trebuie sa \char`\"{}activam\char`\"{} posibilitatea de a desena pe ea. Variabila \hyperlink{namespaceglobal_ae80ab1c7d78e562614d35c3b78e44ea3}{global\+::desenator} este un pointer catre un obiect de tip S\+D\+L\+\_\+\+Renderer, iar pentru a crea efectiv desenatorul chem functia S\+D\+L\+\_\+\+Create\+Renderer care ia ca parametrii fereastra pe care vreau sa desenez si anumite \char`\"{}steaguri\char`\"{} pe care vreau sa le setez. In cazul meu setez S\+D\+L\+\_\+\+R\+E\+N\+D\+E\+R\+E\+R\+\_\+\+A\+C\+C\+E\+L\+E\+R\+A\+T\+ED care va folosi accelerarea hardware daca va putea, ceea ce va imbunatati performanta considerabil.

Mai jos chem T\+T\+F\+\_\+\+Init si Mix\+\_\+\+Open\+Audio care imi ofera posibilitatea de a desena text pe ecran si de e reda sunet. 