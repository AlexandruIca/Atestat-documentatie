Un joc in sine nu are ata de multe aplicatii in lumea reala in afara de entertainment. Totusi consider ca acest proiect reprezinta un foarte bun inceput pentru oricine vrea sa inceapa programarea grafica. Aceasta are mult extrem de multe aplicatii precum vizualizarea organelor corpuli uman, simulari, design etc. \char`\"{}\+Omida\char`\"{} nu este ceva complex dar este o fundatie pentru a dezvolta subiecte mai avansate. Chiar si conceptul scenelor intr-\/un joc este foarte important deoarece cand vrem sa facem un joc nu ne gandim prima data la meniu ci la joc in sine. Dupa ce am terminat jocul intervine problema de reprezentare a meniurilor, conceptul de scena ajutand enorm. Scenele pot fi extinse si reprezentate ca niste arbori pentru flexibilitate maxima(de exemplu submeniuri). De asemenea poate fi extins si modul in care functioneaza \hyperlink{main_8cpp_ae66f6b31b5ad750f1fe042a706a4e3d4}{main}, in loc sa chemam fiecare \hyperlink{classScena_a9e5fcc831ed410b5b2422231ede746ee}{Scena\+::itereaza} cu \hyperlink{classStadiulJocului_a338b9e5467ad6d4d355c3978e04d32d6}{Stadiul\+Jocului\+::timp\+\_\+trecut} variabil, putem stabili un timp maxim (de exemplu 60fps) cu care sa actualizam scena, folositor in cazul unei simulari in care legile fizicii depind foarte mult de timpul trecut. 