\hypertarget{group__group__main__loop}{}\section{Principala repetitie}
\label{group__group__main__loop}\index{Principala repetitie@{Principala repetitie}}
In while-\/ul din functia \hyperlink{main_8cpp_ae66f6b31b5ad750f1fe042a706a4e3d4}{main} se intampla practic tot. Inainte de while sunt inserate doua obiecte de tip \hyperlink{classJoc}{Joc} si \hyperlink{classMeniuStart}{Meniu\+Start}. Scenele sunt folosite ca un fel de stiva, ultima scena fiind cea care trebuie redata. De aceea este inserata prima data scena jocului. Cand vom iesi din meniul de start va fi sters si va ramane scena jocului. Orice scena poate iesi complet din aplicatie daca seteaza \hyperlink{namespaceglobal_a930b1255fa49cd41dc635136822d56ee}{global\+::fereastra\+\_\+inchisa} la fals. O scena va fi stearsa numai daca a fost initializata, si va fi initializata numai daca nu s-\/a intamplat deja. 